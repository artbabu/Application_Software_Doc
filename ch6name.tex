\chapter{Data Store}\label{ch:ch6label}

			Data store component can be considered as repository to store and manage the collection of information used in application. Generally data store can be classified as database in which data is stored in organised manner so that it can be easily accessed, managed and updated. For each application the database is designed according to the business logics and entities such that the data can be interpreted by components for the future use. The database management system(DBMS) is a system software used for creating and managing the databases. DBMS serves as an interface between database and the user or application programs, ensuring that data is consistently organized remains easily accessible. 
			
			Database can be categorised into different types on basis of the  structure of the stored data and the functionality.

\section{Relational Database} 
                 Relation database organises data into one or more tables which is formally described and organised according to the relational model . Each row in the table which is also called as record is a set of values related with each other in some way and a unique key identifying each row. Its focus is to implement simple CRUD - create, Read, update and Delete functionality. Each row represents an object and information about object.A relationship is established among the tables based on the interaction between them  which is derived from business logics of the application. Currently Mysql , PostgreSql and Oracle are the popular RDBMS used for applications.  A connection is established between the application server and the database server using drivers. This    
                  
\begin{description}
  \item[$\bullet$] {\bfseries Mysql :}
  \item[$\bullet$] {\bfseries  :}
  \item[$\bullet$] {\bfseries Mysql :}
\end{description}  
                 
\subsection{Mysql}
\subsection{PostgreSql}
\subsection{SQLite}
\subsection{Oracle}

\section{NoSQL}
\subsection{MongoDB}

\section{Graph Database}

\subsection{Neo4j}
\subsection{Titan}

\section{Mobile Database}

\subsection{Realm}
\subsection{Couchbase}

\section{In - Memory Database}

\subsection{Redis}
\subsection{Aerospike}

\section{Cloud Storage}

\subsection{Amazon S3}
\subsection{Amazon EBS}
\subsection{Google Cloud Storage}
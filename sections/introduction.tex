\chapter{Introduction}\label{ch:introduction}
			  			Application	Software is program or group of program customised to perform group of activities for end user. Generally software are classified as system software and application software.System software consists of low-level programs that is designed to run computer's hardware and application programs like compilers, loaders, linkers and so on. Application software resides above system software like database programs, word processors, spreadsheets. In this survey, we focus on the architecture and process of development of different types of application software.
			
			Technology has come long way since the first computers were created, back around the start of World War II. The first generation of software application are command line programs which are mostly single command at a time and uses it to accomplish all the application requirements in that particular loop. These programs are shared as binaries(executables) which can be compiled from the source code, specific to the architecture. 


\section{Examples}
You can also have examples in your document such as in example~\ref{ex:simple_example}.
\begin{example}{An Example of an Example}
  \label{ex:simple_example}
  Here is an example with some math
  \begin{equation}
    0 = \exp(i\pi)+1\ .
  \end{equation}
  You can adjust the colour and the line width in the {\tt macros.tex} file.
\end{example}

\section{How Does Sections, Subsections, and Subsections Look?}
Well, like this
\subsection{This is a Subsection}
and this
\subsubsection{This is a Subsubsection}
and this.

\paragraph{A Paragraph}
You can also use paragraph titles which look like this.

\subparagraph{A Subparagraph} Moreover, you can also use subparagraph titles which look like this\todo{Is it possible to add a subsubparagraph?}. They have a small indentation as opposed to the paragraph titles.

\todo[inline,color=green]{I think that a summary of this exciting chapter should be added.}
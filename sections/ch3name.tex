\chapter{Presentation Layer}\label{ch:ch3label}

			The design of presentation layer begins by identifying the potential customer base, and understanding the objectives of the customer and task the user wish to accomplish when using the application. As the main focus of the application design must be user centred, the sequencing of tasks or operations should be designed as per the user expectation. It may be a structured step-by-step user experience or unstructured experience where they can perform more than one tasks simultaneously. One important aspect that has great influence on the choice of technology is functionality required for the user interface. Prioritizing the requirements like rich functionality, user interaction, user responsiveness , user interface , personalization requirements and graphical support will help to choose User Interface Type.
			
			Most of the user interface requirements needs more than one UI type.The frameworks, tool and implementation language used for UI requirements differ according  to the application platform.
			
\section{Mobile Application}
         
            	Mobile application can be classified as thin client and rich client application.Rich client is designed to support the native or hybrid applications which supports offline or intermittent online scenarios. Web or thin clients supports only online  scenarios.Resources constraint should e taken into account while designing user interface for mobile application.
            	
            	Many developers or vendors want to write an application once using single codebase , the run that application in multiple platform with a very small change in codebase. This approach is famously known as Write Once, Run Anywhere (WORA).WORA removes the redundancy in development at the cost of user experience and application performance.This application is called as hybrid mobile application which has native application experience. Improving the performance and user experience are in future roadmap.     
            	
\begin{figure}[!htb]
  \includegraphics[width=\linewidth]{figures/nativeMobApp.png}
	 \caption{Languages - Native Moblie Application }
  \label{fig: Languages - Native Moblie Application}
\end{figure}		            	       	

\subsection{Apache Cordova - Frameworks}
              
              Apache Cordova is an open-source mobile cross-platform development framework which allows you to use standard web technologies - HTML5, CSS3, and JavaScript . In mobile development, this framework extends more than one platform avoiding the implementation time taken for each platform.Applications execute within wrappers targeted to each platform, and rely on standards-compliant API bindings to access each device's capabilities such as sensors, data, network status, etc. It is flexible in developing application which has combination of native application components with a web view that can access device level APIs using Core Plugins.           
            	
\subsection{Ionic - Frameworks}		
              
              Ionic is a free open source framework facilitates the development of hybrid native mobile using web technologies like HTML,Javascript and CSS. Its core is built up with AngularJS javascript framework and supports SASS (Syntactically Awesome Style Sheets) CSS extension . This features qualifies Ionic as a unique framework.It simplifies development and testing of the application by providing client side model view controller architecture. User Interface response is pretty faster than Backbone and Knockout and it possesses a large number of 3rd party plug ins and extensions. Currently Ionic supports iOS and Android devices, Windows phones and FirefoxOS support are considered to be in future roadmap. This framework can be only used for hybrid mobile application.Ionic is built on top of Apache Cordova which is open source mobile development framework.
              

\begin{figure}[!htb]
  \includegraphics[width=\linewidth]{figures/hybridMobApp.png}
	 \caption{Cordova Languages - Hybrid Moblie Application }
  \label{fig: Cordova Languages - Hybrid Moblie Application}
\end{figure}		 

   
\subsection{Mobile Angular UI - Frameworks}     
			
			Mobile Angular UI is an HTML 5 framework which uses bootstrap 3 and AngularJS with better user response. In other words it combines the best of the web framework which enables you to create hybrid application and web application. This framework works well with all browsers including older versions and offers feature of customizing build workflow with GRUNT. Grunt is used run automation  using a task runner with zero effort.
			
\subsection{Xamarin - Frameworks}

				Xamarin applications are written in C\# and compiled to binaries compatible with native device giving access to all the device API through C\# wrappers.To develop application that look like native, Xamarin uses native UI mechanisms for each platform. Recently Xamarin.Forms was released which enables the use of common UI technology, XAML and work on cross platforms using native controls. Xamarin takes advantage of all the third party libraries written for .NET stack 

\begin{figure}[!htb]
  \includegraphics[width=\linewidth]{figures/XamarinhybridApp.png}
	 \caption{ Xamarin Languages - Hybrid Moblie Application }
  \label{fig: Xamarin Languages - Hybrid Moblie Application}
\end{figure}							
			



\section{Web Application}

				Server-side HTML,  and Service-oriented single-page Web apps are marked as a three generations in evolution of web. Web 1.0 is the first stage of world wide linking web pages and hyperlinks through URL.It was used as information portal where there server  as it was read only Web. The server generates HTML-content and sends it to the client as a full fledged HTML-page.It had limited user interactions. Web 2.0 was introduced to promote read-write web which contains user generated content and improved the more user interaction. Web applications used Ajax programming which uses javascript to upload and download new data from the web server without full page reload. Some applications where programmed in Flex which helped to populate huge amount of data and other heavy user interactions. The application programmed in Flex are compiled and displayed as Flash within the browser. Then the web which is capable of understanding the content and context was introduced. It was famously known as Semantic Web .Applications are 

\section{Desktop Application}


\section{General Tools}

\subsection{ReactiveX - libraries}

			Tremendous increase in need to respond and trigger actions based on data events in mobile application expanded the requirement of handling events in asynchronous manner. Component Object model was one of the previous technology that was used to execute asynchronous tasks. Recently Netflix introduced ReactiveX library which to process asynchronous task with robust architecture. ReactiveX is a library used for developing asynchronous  and event based programs using observable sequence.This observer pattern promotes more real time experiences of the mobile or web application. Even modifying a single field in the current page triggers a instant save to back end, for example Twitter follow feature enabled for some profile can be immediately reflected to other connected user and so forth.
			
\subsection{picasso - librares}

\subsection{HTML 5 - Languages}
\subsection{Swift - Languages}
\subsection{Java - Language} 
\chapter{Presentation Layer}\label{ch:ch3label}

			The design of presentation layer begins by identifying the potential customer base, and understanding the objectives of the customer and task the user wish to accomplish when using the application. As the main focus of the application design must be user centred, the sequencing of tasks or operations should be designed as per the user expectation. It may be a structured step-by-step user experience or unstructured experience where they can perform more than one tasks simultaneously. One important aspect that has great influence on the choice of technology is functionality required for the user interface. Prioritizing the requirements like rich functionality, user interaction, user responsiveness , user interface , personalization requirements and graphical support will help to choose User Interface Type.
			
			Most of the user interface requirements needs more than one UI type.The frameworks, tool and implementation language used for UI requirements differ according  to the application platform.
			
\section{Mobile Application}
         
            	Mobile application can be classified as thin client and rich client application.Rich client is designed to support the native or hybrid applications which supports offline or intermittent online scenarios. Web or thin clients supports only online  scenarios.Resources constraint should e taken into account while designing user interface for mobile application.
            	
\subsection{Ionic - Frameworks}		
              
              Ionic is a free open source framework facilitates the development of hybrid native mobile using web technologies like HTML,Javascript and CSS. Its core is built up with AngularJS javascript framework and supports SASS (Syntactically Awesome Style Sheets) CSS extension . This features qualifies Ionic as a unique framework.It simplifies development and testing of the application by providing client side model view controller architecture. User Interface response is pretty faster than Backbone and Knockout and it possesses a large number of 3rd party plug ins and extensions. Currently Ionic supports iOS and Android devices, Windows phones and FirefoxOS support are considered to be in future roadmap. This framework can be only used for hybrid mobile application.Ionic is built on top of Apache Cordova which is open source mobile development framework.
              
\subsection{Apache Cordova - Frameworks}
              
              Apache Cordova is an open-source mobile cross-platform development framework which allows you to use standard web technologies - HTML5, CSS3, and JavaScript . In mobile development, this framework extends more than one platform avoiding the implementation time taken for each platform.Applications execute within wrappers targeted to each platform, and rely on standards-compliant API bindings to access each device's capabilities such as sensors, data, network status, etc. It is flexible in developing application which has combination of native application components with a web view that can access device level APIs using Core Plugins.           
   
\subsection{Mobile Angular UI - Frameworks}     
			
			Mobile Angular UI is an HTML 5 framework which uses bootstrap 3 and AngularJS with better user response. In other words it combines the best of the web framework which enables you to create hybrid application and web application. This framework works well with all browsers including older versions and offers feature of customizing build workflow with GRUNT. Grunt is used run automation  using a task runner with zero effort.
			
\subsection{Xamarin - Frameworks}

						
			

\subsection{volley - libraries}
\subsection{picasso - librares}

\subsection{HTML 5 - Languages}
\subsection{Swift - Languages}
\subsection{Java - Languages}
           	
\section{Web Application}
		
\subsection{Bootstrap - Frameworks}
\subsection{AngularJS - Framework}
\subsection{Foundation - Framework}

\subsection{Javascript - Languages}

\subsection{JQuery - Libaries}
    
\section{Desktop Application}        

\subsection{QT4 - Frameworks}
\subsection{gtk - Frameworks}

\subsection{Java FX - Libaries}	

\subsection{Java - Languages}
            	 
			
			
			
			
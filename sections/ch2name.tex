\chapter{Software Application Architecture}\label{ch:ch2label}
						
						Overall structure of the software application can be represented in the logical grouping of components into layers that interact with each other according to the functionality and features of the application. This chapter will focus on how the application is divided into components and the services provided by each layer.These layers help to uniquely identify the different kind of task performed by each components. Each layer would consist of multiple sub layers which performs specific type of task which would greatly help while debugging.

\begin{figure}[!htb]
  \includegraphics[width=\linewidth]{figures/SoftwareArch.png}
	 \caption{Software Application Architecture}
  \label{fig: Software Application Architecture}
\end{figure}						
					
						By analysing the types of components that exist in most solutions, you can construct a meaningful map of an application or service, and then use this map as a blueprint for your design. Dividing an application into separate layers that have distinct roles and functionalities helps you to maximize maintainability of the code, optimize the way that the application works when deployed in different ways.Figure 2.1 represents the highest and most abstract level of the software architecture.It gives clear perspective of theirs interaction with users, relationship with other application that requests the services implemented within the application business layer,web services hosted by the application , data sources and the remote service used by the applications offered by other software.

\section{Basic Architecture Layer}
						
						The common three layer design consist of Presentation Layer ,Business Layer and Data Layer. 
						
						
\begin{description}
  \item[$\bullet$] {\bfseries Presentation Layer :} This layer consist of two sub components which contains purely user based functionality that manages the user interaction with the application. This layer acts as a bridge between the user and business logic layer. User Interface and user interface logic are the two major components found in this layer.This layer is responsible for the user input and display in addition to the components that organise user Interaction.To make the code modular the implementation of visual elements which would display data to user and accepts input is separated from the presentation logic which gives rise to the idea of model, view and controller.To handle the design based issues , security issues and improve the application performance and user interface responsiveness, the extra components are introduced which discussed in forth coming chapters.
  \item[$\bullet$] {\bfseries Business Layer :} This layer implements core functionality of the application and is concerned with the processing, transformation, retrieval and management of data.The core implementation include the business rules,entities and work flow of the rules. Application façade is a optional component which acts a simplified wrapper on business logic components by combining multiple operations into a single operation in a logical way. This reduces dependencies and improves modularity. This prevents the external service request from knowing the details of the business components and the inter relationship between the operations.
  \item[$\bullet$] {\bfseries Data Layer :} Data access logic and data store are the major components of this layer. Much more components like object document mapper , object relational Mapper etc which deals with translation of data can be introduced into this layer based on the requirements.This layer provides access to data within the boundaries of the system and the data exposed by the other networked systems through web service.Data access logic components forms a interface that the components in the business layer consumes. 
\end{description}			

\section{ Requirement-Based Components  }

			Some layers are pluggable in the architecture on requirement basis.When an application must provide service to external system , the service layer is used to expose the business functionality without exposing the operation signature.There are components which are added to the architecture to handle issues.The issues and the corresponding layer which handles the issue can be categorised based on the architecture layers.
			
\subsection{Issues and Add-on - Presentation Layer}	

\begin{description}
  \item[$\bullet$]{\bfseries Caching :} Caching is considered to be a best mechanism to improve application performance and user interface responsiveness. Caching is used in presentation layer to avoid frequent data lookups which in turn avoids network round trips and to store the result of the repetitive process to avoid the unnecessary duplicated processing. Its important to make the cache thread-safe when it is used in multi-thread environment.
  \item[$\bullet$]{\bfseries Exception Management :} Employing a centralised exception management is considered to be a consistent way to manage the unexpected exceptions. The exception which propagates across layers are blocking issue which would crash the application. Differentiating the error occurred into system-based or business logic based will be helpful to provide a friendly error message to the users. If it is business errors,a error message can be displayed and allow user to re-try that particular operation. In case of system error, a error message can be displayed along with the troubleshooting assistance. It is important to ensure no sensitive data is exposed in the error pages, error message and log files.
  \item[$\bullet$]{\bfseries Compositions :} User interface composition patterns and templates supports the creation of the views and the presentation layout at the run times.It is easy to develop and maintain if the presentation layers uses independent modules and views that are composed at run time. These templates helps to minimise the code and library dependencies that would otherwise force recompilation and redeployment of the modules when the dependencies are upgraded. 
   
\end{description}  		

\subsection{Issues and Add-on - Business Layer}	
  		The business layer includes the previous add-ons like caching 
  		



Data Layer
Batching : To achieve high performance , the similar queries are batched avoiding multiple database request and execution. Each database request is considered to be a overhead. Batched commands reduces the round trip time to database and minimizes network traffic as well as the size of the connection pool.


\begin{description}
  \item[$\bullet$]{\bfseries Authentication : } Security and reliability of an application is important for business layer where authentication comes into picture.If your business layer will be used by multiple application , plugging an  authentication mechanism  into the business layer ensures the security of the sensitive data. In case of client server application, the application layer lies in client and the business layer lies in server application. Both application communicates over network since they are deployed in different machines. In this case, a centralised platform which handles authentication is mandatory for the reliability of the application
  \item[$\bullet$]{\bfseries Authorization :} In most of the applications the authorization components manages issues like information disclosure and user privileges for specific activity.Authorization are used to allow specific group of user to perform some sensitive business information and view some sensitive data. It promotes hierarchy of user privileges and the related operations ensuring security of the data sources.It is important to implement the authorization code in a modular way which doesn’t  impact on performance of the application and doesn’t have any dependency with business layer .
  \item[$\bullet$]{\bfseries Logging and Auditing :}  Maintaining a centralised logging and auditing system would be useful in tracking the issues in production and the illegal user actions in important situations. System Monitoring tools are used to identify state and performance of the application. Logging helps to track this performance and the load handled by the application. It is important to ensure that logging failure doesn’t affect the functionalities of business layer.   
\end{description}  

\subsection{Issues and Add-on - Data Layer}	

\begin{description}
  \item[$\bullet$]{\bfseries Batching :}  To achieve high performance , the similar queries are batched avoiding multiple database request and execution. Each database request is considered to be a overhead. Batched commands reduces the round trip time to database and minimizes network traffic as well as the size of the connection pool.
 \item[$\bullet$]{\bfseries Connection :} Establishing connection to data sources is considered to be basic part of data access layer.Creating and maintaining data sources connection  in inefficient way directly impacts the performance of the applications. Connection pooling within a security model would serve the purpose in efficient way.Holding connection for short period and avoiding the usage of user DSN or system DSN would help to achieve high performance and security. 
 \item[$\bullet$]{\bfseries Object Relational Mapping (ORM) :} When the application is designed using object oriented model, It is difficult to translate the relational model to business entities. ORM can be used to create schema which supports the object model and provides a mapping between the database and domain entities.When working with Web applications or services, group entities and support options that will partially load domain entities with only the required data—a process usually referred to as lazy loading. This allows applications to handle the higher user load required to support stateless operations, and limit the use of resources by avoiding holding initialized domain models for each user in memory.
\end{description}  

\subsection{Issues and Add-on - Service Layer}	
                        In case of external application acting as service consumer or the application is designed in client server model, the service layer acts as interface between the systems. Service Interface represent the operations supported and their associated parameters and data transfer objects. To avoid multiple call over network, the service interface are designed to serve multiple operations grouped based on business logics. Authentication is used to determine the identity of the service consumer.Authorization is used to determine which resources or actions can be accessed/possessed by an authenticated service consumer. Communication component can be designed based on the requirement like request-response ,duplex communication and asynchronous communication.Message queueing can be used to handle tremendous burst of asynchronous message .
 